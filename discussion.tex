\chapter{Discussion}
\label{chp.discussion}

%a. Summarize results briefly.
%b. Discuss results in non-statistical terms. Answer the research question and

In this chapter, we first conclude the work in Section~\ref{sec.discussion.conclusion}, thereafter 
highlight some future works
in Section~\ref{sec.discussion.furtherwork}.

\section{Conclusion}\label{sec.discussion.conclusion}
%a. Organize this section with headings
%b. Explicitly discuss the implications of the results. Integrate your results with the theoretical background and very relevant literature findings.
%e. It is appropriate to speculate on the meaning of the results as long as it is made explicit that that is what the writer is doing. 

In this study, the ability of inhibitory control of the 
mothers and of the non-mothers are 
compared in an extended Stroop task, 
which is designed to introduce the emotional
interferences, particularly to include the ones from the babies,
enabling the interaction between a mother and 
the emotional facial expression from a baby.
In addition, as a part of the 
parenthood study project\footnote{\url{https://www.psychologie.hu-berlin.de/de/prof/bio/forschung/forschungsprojekte}},
a significant amount of data for the extended Stroop task
is collected, based on which three hypothesizes are examined,
leading to following conclusions.

\begin{itemize}
\item[C1] 
There is no evidence to support that
a mother or non-mother would behave differently when mixing up 
two kinds of interfering facial expressions (namely, the ones from a baby 
and from an adult).

\item[C2] 
The data does not support that facial expressions from a baby could trigger larger interferences compared to the ones from an adult, 
when without distinguishing the mothers and the non-mothers.


% although the facial expressions from babies are found generally more attractive than the ones from adults\citep{lorenz1943angeborenen}. Instead, the negative facial expressions of adults are found to have a larger impact compared to the ones of adults, upon the ability of inhibitory control, which may due to that the negative facial expressions of adults indicate more social cues, trigger larger reflection of emotions, and strongly require the response compared to the ones of babies.

\item[C3] 
When co-considering the factors from the mothers and non-mothers,
as well as from the different kinds of facial expressions,
the non-mothers are found more prone to be interfered by the facial expressions
from an adult, whereas the mothers behave similarly
in front of both. 
Meanwhile, the mothers are interfered more by a positive (smiling)
baby face.

\end{itemize}

Overall,
we conjecture that the motherhood does influence
the ability of the inhibitory control, in the sense that
the mothers behave differently from the non-mothers
when being interfered by an expression from an adult or from a baby.
In addition, 
the mothers tend to be interfered more by a positive baby face, which may 
stem from the behavior-tuning in the daily baby-sitting or from 
the biological changes, and is left for the future works.


\section{Future Work}\label{sec.discussion.furtherwork}

There exist limitations that require some further efforts to fix in the next step. For example,
due to the small amount of participants, some effects are hard to conclude. Thereby, collecting more data is beneficial for the further study.
% Moreover, although the preparation of the EMG is properly and without any bubbles, the impedance of EMG electrodes varied a lot in practice,
% which may still cause malformed signals. 
% \todo{what do you want to say? which may lead to the artifacts of the recording.}
Besides, according to the feedbacks from the participants, the angry facial expression from the babies are rare to see (and look cute), making them amusing, leading to false records. 
Beyond these, some suggestions for the future works are listed in the following.

% limitations are also considered 
% as potential improvements to the current work,
% which, however, 
% are mainly due to 
% the limited conditions.
%and are non-trivial to address.

\begin{itemize}
\item[F1]
In this study, the facial expressions from 
stranger babies are displayed, which may interfere differently
to a mother than her own baby. 
As future works, 
the facial expressions from both a stranger baby and 
from the participant's baby are desired, introducing more 
fine-grained interfering configurations.
\item[F2]
Due to the limitations in the data collection,
the groups of mother 
and non-mother are collected from different persons.
To remove the influences coming from the different individuals,
ideally, paired data should serve the study better, namely,
the same task should be fulfilled by the same female before and after she
has a baby, which, however,
may be difficult in practice.
\item[F3]
the mothers are found to be interfered more by a positive baby face, which may due to the biological changes, as stated in the literature. 
Meanwhile, as mentioned, 
another possible explanation is that the experience of baby-sitting makes a mother more sensitive to certain reactions from a baby, which may require further explorations. 
For example, one could introduce a particular group of non-mothers who also conduct baby-sitting, like a nursery teacher,
thereby clarifying whether the observed differences come from the  baby-sitting experiences in daily life.

% For example, comparing the ability of inhibitory control between groups of non-mothers with different levels of exposure to the daily care of young infants. (e.g. nursery workers or teachers as compared to those with no experience of childcare), or between fathers and non-fathers in the future work.
\end{itemize}



%A limitation is a weakness or handicap that potentially limits the internal or external validity of the results, such as using a sample with a particular characteristic such as all males. Most limitations should have been considered when the study was conceptualized. Therefore, limitations in this section are those that were largely outside the control of the researcher.
%b. Often limitations include a statement of the generalizability of the results, controls that may be impossible to meet, etc. For example, if you must use intact groups rather than random assignment, how might this affect the interpretation of your results? 




% \begin{itemize}
% \item[-] 
% Based on the designed extended Stroop task, 
% there actually exist another kind of interference,
% namely, the participant has to prevent themselves from mirroring 
% the displayed facial expressions, differing from the ones 
% discussed where the participants overcome the distraction 
% from the facial expression to concentrate on the word instruction.
% \todo{Are you sure?}
% To achieve this, novel mechanism is desired to be implemented.

% % analyze
% % another kind of interference,
% % where the participant could be asked to 

% % In the study, The response time refers to the latency between the time when the captured signals exceed the threshold for the first time and the reference point, during which, two mechanisms are conducted, one is the ability of resisting the perception of irrelevant facial expression in the background, the other is the ability of resisting the production of similar emotional expression according to the facial expressions. With the combination of the electroencephalography, these two processes could be separately investigated.
	
% \item[(2)] 
 

% % Since the hormones and the biological changing could be one of the reason for such effect. The other potential explanation is also worth investigation, such as the experience of childcare. It will be important in future to investigate whether there are differences in the inhibitory control ability between groups of non-mothers with different levels of exposure to the daily care of young infants. (e.g. nursery workers or teachers as compared to those with no experience of childcare). Another important follow-up study will be to compare responses of fathers and non-fathers. Such studies may help us to further delineate whether differences in infant face processing between those with and without children is specific to motherhood or relates to the experience of parenting more generally.
	
% \item[(3)] 
% Due to the limitations in collecting the data,
% the data employed in this study is 
% static in the sense that the groups of mother 
% and non-mother are constituted of different people.
% However, to remove the influences coming from the different individuals,
% ideally, paired data might serve the study better, namely,
% the same tasks are conducted by the same female before and after she
% has a baby.

% % fixed per the time of the experiments.

% % It should also be noted that the data presented in this study are cross-sectional and the mothers who take part in the study, have a child aged from two to six months. Future studies may wish to investigate whether the Stroop effect towards infant faces changes from non-parent, through pregnancy, to becoming a first-time parent. 
	
% \item[(4)]

% % A further limitation is that the current design used pictures of unfamiliar infants. It will be important for future studies to explore how such inhibitory control may vary in relation to a mother's own child.
	
% \end{itemize}

%\item[(5)] Mothers are more likely to grow tense around foreign faces. This effect may be linked to the fact that the body is immunosuppressed during those early months, and “outsiders” might harbour germs that endanger the fetus.

%a. Provide specific guidance based on the dissertation finds and they relate to the extant theoretical and empirical base.
%b. Why is the proposed research needed and what form should it take. 

