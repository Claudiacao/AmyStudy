\chapter{Task Design and Hypothesis}
\label{chp.hypothesis}


In this chapter, we first introduce the extended Stroop task in Section~\ref{sec.hypothesis.expstrooptask} 
which is employed as a major 
instrument to explore the relationship between the ability of inhibitory control and the motherhood;
thereafter, the experiment design and three hypotheses are introduced in Section~\ref{sec.hypothesis.design} and~\ref{sec.hypothesis}.

\section{Extended Stroop Task}\label{sec.hypothesis.expstrooptask}
Recall that in the traditional Stroop task, 
as introduced in Section~\ref{subsec.measureability},
the participants are asked to name the printing color
of a word, whose literal notion (of a color)
serves as the interferences, where the Stroop effect indicates the phenomenon that a longer response time and a higher error rates are observed 
given a word with an incongruent literal notation
relative to its printing color~\citep{stroop1935studies}.
Such Stroop effect could be explained by the automatic theory as summarized in Section~\ref{subsec.measureability}, namely,
participants could automatically understand
the meaning of a word as a result of habitual reading,
whereas 
the recognition of the interfering printing color
might require a hesitation due to that the recognition 
 is not an ``automatic process''~\citep{monahan2001coloring}. 
In addition,
the traditional Stroop task has been extended 
to investigate the effects of emotional interference,
leading to the emotional Stroop task. 
Therein 
participants are asked to name the printing color of a word,
which literally denotes an emotional concept like ``hate''.
\citeauthor{algom2004rational}
argues that, however, the incongruent semantic meaning is 
missing in the emotional Stroop task, which lies at the heart of the 
original Stroop task,
making the observed effect different from a Stroop effect~\citep{algom2004rational}.


In this work,
to incorporate the emotional interference in a Stroop task,
inspired by the emotional interference tasks~\citep{otte2011interference,lee2007controlling}, 
an extended Stoop task is proposed.
According to~\citep{dimberg2000unconscious,lundqvist1995facial,wild2001emotions},
the procedure to simulate a facial expression (with a facial expression)
is also ``automatic process'',
whereas a facial expression in response to a word is not habitual.
Consequently,
corresponding to the printing color of a word and its literal notation 
from the original Stroop task respectively,
in the extended Stroop task,
we ask a participant to make a facial expression according to a word, which is 
displayed together with an interfering picture including an emotional facial expression. 
In particular, a participant is asked to simile or frown 
in response to a word, namely, 
either ``happy'' or ``angry'', 
meanwhile a photo including a happy or angry facial expression with 
the same or different emotion, 
denoted as ``positive'' and ``negative'' respectively, 
serves as interference. 
To further investigate the impact of the interaction between a mother and a baby over the inhibitory control,
the distracting facial expressions from 
both adults and babies are introduced.
For example, as displayed in Figure~\ref{fig.method.task},
a word ``happy'' as well as an angry facial expression from a baby are displayed, 
where a smile is expected from the participants 
as a correct response
according to the word ``happy''.

% this is for method chapter
% Electromyography(EMG), an experimental technique for evaluating and recording the electrical activity produced by skeletal muscles, becoming a standard tool in the study of the production of the facial expressions. Akin to \citeauthor{otte2011interference} as well as to \citeauthor{lee2007controlling}, Electromyography (EMG) is employed in the extended Stroop Task to measure the correctness and the response time (RT) of the facial expression from a subject by capturing the movements of two groups of facial muscles, namely, the zygomatic major muscle controlling smiling and the corrugator supercilii muscle controlling frowning \citep{otte2011interference,lee2007controlling}.


\section{Experimental Deign}\label{sec.hypothesis.design}
\begin{table}[!t]
\centering
% change the vertical space, 1.0 is the standard 
\ra{1.7}
\caption{Experiment conditions for the group of mother ($m1\cdots m8$,)  and non-mother ($n1\cdots n8$ ), indicating the number for the 8 different conditions in each group.}
\label{tab.hypothesis.whole.design}
% change the overall size 
\resizebox{0.7\textwidth}{!}{
% @{} remove the space at two sides
\begin{tabular}{@{}ccc|cc@{}}
\toprule
\multicolumn{3}{c|}{experiment conditions}& adults' faces &babies' faces\\
\midrule
\multirow{4}{*}{mother}&\multirow{2}{*}{positive}&``Freude''&$M_1$&$M_2$\\
&&``Ärger''&$M_3$&$M_4$\\
% \cmidrule{2-5}
&\multirow{2}{*}{negative}
&``Freude''&$M_5$&$M_6$\\
&&``Ärger''&$M_7$&$M_8$\\
\cmidrule{1-5}
\cmidrule{1-5}
\multirow{4}{*}{non-mother}&\multirow{2}{*}{positive}&``Freude''&$N_1$&$N_2$\\
&&``Ärger''&$N_3$&$N_4$\\
% \cmidrule{2-5}
&\multirow{2}{*}{negative}
&``Freude''&$N_5$&$N_6$\\
&&``Ärger''&$N_7$&$N_8$\\
\bottomrule
% \multirow{4}{c}{the group of non-mother}\\
\end{tabular}
}
\end{table}
In this work,
to examine the differences in the ability of inhibitory control between mothers and non-mothers, 
the responses from the participants to the German word \textit{Freude} (happy) or \textit{Ärger} (angry)
are recorded
when being distracted by a facial picture from 
a \textit{baby} or an \textit{adult} with a \textit{positive} or a \textit{negative} facial expression.
The responses are further examined in terms of the response time 
and the accuracy where a response is correct if it follows the displayed word and vice versa as indicated in Figure~\ref{fig.method.task}.
In particular,
the identity of a facial picture (namely, an adult or a baby),
the desired response indicated by the words (namely, ``Freude'' or ``Ärger''),
and the emotion expressed on the interfering facial picture (namely, a positive or negative emotion) are independent variables,
leading to eight different experiment conditions;
meanwhile,
the motherhood (namely, a mother or a non-mother)
is another independent variable, enabling the comparison.
Given one of the eight experiment conditions,
the response from a participant is recorded in terms of  
the response time and the accuracy, which are the dependent 
variables to quantify the ability of the inhibitory control,
constituting an \textit{experiment trial}.
According to the experiment setting,
the identification of the participant (namely, either a mother or a non-mother),
all trials are sub-clustered into sixteen subsets as summarized in 
Table~\ref{tab.hypothesis.whole.design}.
At the end of this section, 
to facilitate our descriptions,
several notations are defined in the following.


\textbf{Notation.}
Individual trials in our experiments are denoted as $t_i\in T$, 
where $T$ represents the set of all trials. 
As defined in Table~\ref{tab.hypothesis.whole.design}, 
the sixteen subsets of $T$ are $M_1\cdots M_8$ for the group of mother 
and $N_1\cdots N_8$ for the group of non-mother.
Furthermore, given a subset $T^\prime\subset T$, the average accuracy and the average response time among all $t_i \in T^\prime$ are defined as $\mathit{accuracy}(T^\prime)$ and $\mathit{rtime}(T^\prime)$ respectively. 
I further denote several subsets of $T$: they are 
$T_{mc}$, $T_{mi}$, $T_{nc}$, and $T_{ni}$, as defined in Table~\ref{tab.hypothesis.1},
which will be used in hypothesis 1;
$T_{bc}$, $T_{bi}$, $T_{ac}$, and $T_{ai}$ in Table~\ref{tab.hypothesis.2} for hypothesis 2;
and $T_{mbc}$, $T_{mbi}$, $T_{mac}$, $T_{mai}$, $T_{nbc}$, $T_{nbi}$, $T_{nac}$, and $T_{nai}$ for 
hypothesis 3. The meaning of these subsets are introduced in the corresponding tables and hypotheses.

		
\section{Hypotheses}\label{sec.hypothesis}
To investigate the relationship between the motherhood and the ability of the inhibitory control,
a research question to answer is that 
whether the motherhood makes one more prone to the interferences in terms of emotional facial expressions.
According to this research question, three hypotheses are stated, depending on whether the interfering facial expressions come from
an adult or a baby.
\begin{itemize}
\item[(a)] When comparing a mother with a non-mother, a mother is more prone to be distracted by the interferences in terms of emotional facial expressions.
\item[(b)] When distinguishing the source of the interferences, the interfering facial expressions from a baby and from an adult impact differently, disregarding of the status of motherhood.
\item[(c)] 
When further considering the interaction between a mother and an interfering facial expression from a baby, 
compared to a non-mother, a mother is more likely to be impacted by the emotional facial expressions from a baby.
\end{itemize}


Actually, to examine these three hypotheses, 
one could simply control some independent variables and compare the Stroop effects between (among) different groups,
namely, for hypothesis (a), the mother and non-mother is the controlled independent 
variable; similarly, to examine hypothesis (b), the source of the interfering facial expression,
namely, a baby or an adult, needs to be controlled; and for (c),
both mentioned independent variables should be controlled.

In addition,
recall that, as described in Section~\ref{sec.hypothesis.expstrooptask},
in the extended Stroop Task, the Stroop effects are 
in terms of the different responses when facing congruent or incongruent interference,
namely, 
a shorter response time and a higher accuracy 
in response to a congruent pair of word and facial picture, meanwhile a longer response time and a lower accuracy 
in response to a incongruent pair of word and facial picture.
Therefore, akin to~\citep{bugg2008multiple}, in this work,
the Stroop effect is quantified in terms of the differences 
of the response time and of the accuracy between congruent and incongruent trials.
Accordingly, three hypotheses in the context of the extended Stroop task are derived, using the notations
from Section~\ref{sec.hypothesis.design}.
\begin{table}[!t]
\centering
\ra{1.7}
\caption{Subsets defined for hypothesis 1. }
\resizebox{0.7\textwidth}{!}{
\label{tab.hypothesis.1}
\begin{tabular}{@{}cccc@{}}
\toprule
\multicolumn{2}{c}{experiment conditions}&subsets\\
\midrule
\multirow{2}{*}{mother}&congruent&$T_{mc}=M_1\cup M_2 \cup M_7 \cup M_8$\\
&incongruent&$T_{mi}=M_3\cup M_4 \cup M_5 \cup M_6$\\
\cmidrule{2-3}
\cmidrule{2-3}
\multirow{2}{*}{non-mother}&congruent&$T_{nc}=N_1\cup N_2 \cup N_7 \cup N_8$\\
&incongruent&$T_{ni}=N_3\cup N_4 \cup N_5 \cup N_6$\\
\bottomrule
% \multirow{4}{c}{the group of non-mother}\\
\end{tabular}
}
\end{table}
 
\noindent - \textbf{H1}:
As indicated in Table~\ref{tab.hypothesis.1}, 
compared with the group of non-mother ($T_{n.}$), 
the group of mothers ($T_{m.}$) are more prone to the interferences, resulting in a larger Stroop effect in the group of mothers than of the non-mothers, as formally summarized in Equation~\ref{eq.accuracy.h1} and~\ref{eq.rtime.h1}.

\begin{align}
\begin{split}\label{eq.accuracy.h1}
\mathit{accuracy}(T_{mc})-\mathit{accuracy}(T_{mi})&>
\mathit{accuracy}(T_{nc})-\mathit{accuracy}(T_{ni})\\
&\mathit{AND}\\
\end{split}\\
\begin{split}\label{eq.rtime.h1}
\mathit{rtime}(T_{mc})-\mathit{rtime}(T_{mi})<
\mathit{rtime}(T_{nc})-\mathit{rtime}(T_{ni})\\
\end{split}
\end{align}


\begin{table}
\centering
\ra{1.7}
\caption{Subsets defined for hypothesis 2.}
\label{tab.hypothesis.2}
\resizebox{0.7\textwidth}{!}{
\begin{tabular}{@{}cccc@{}}
\toprule
\multicolumn{2}{c}{experiment conditions}&subsets\\
\midrule
\multirow{2}{*}{babies'}&congruent&$T_{bc}=M_2\cup M_8\cup N_2\cup N_8$\\
&incongruent&$T_{bi}=M_4 \cup M_6 \cup N_4 \cup N_6$\\
\cmidrule{2-3}
\cmidrule{2-3}
\multirow{2}{*}{adults'}&incongruent&$T_{ac}=M_1 \cup M_7 \cup N_1\cup N_7 $\\
&incongruent&$T_{ai}=M_3\cup M_5 \cup N_3 \cup N_5$\\
\bottomrule
\end{tabular}
}
\end{table}


\noindent - \textbf{H2}:
As denoted in Table~\ref{tab.hypothesis.2},
compared with an interfering facial expression from an adult ($T_{a.}$),
a facial expression from a baby ($T_{b.}$) could lead to
a larger Stroop effect, as formally indicated in Equation~\ref{eq.accuracy.h2} and~\ref{eq.rtime.h2}.

\begin{align}
\begin{split}\label{eq.accuracy.h2}
\mathit{accuracy}(T_{bc})-\mathit{accuracy}(T_{bi})&<
\mathit{accuracy}(T_{ac})-\mathit{accuracy}(T_{ai})\\
&\mathit{AND}\\
\end{split}\\
\begin{split}\label{eq.rtime.h2}
\mathit{rtime}(T_{bc})-\mathit{rtime}(T_{bi})>
\mathit{rtime}(T_{ac})-\mathit{rtime}(T_{ai})\\
\end{split}
\end{align}

\begin{table}[!t]
\centering
\ra{1.7}
\caption{Subsets defined for hypothesis 3.}
\label{tab.hypothesis.3}
\resizebox{0.7\textwidth}{!}{
\begin{tabular}{@{}ccc|c@{}}
\toprule
\multicolumn{3}{c|}{experiment conditions}& subsets\\
\midrule
\multirow{4}{*}{mother}&\multirow{2}{*}{babies'}&congruent&$T_{mbc}=M_2\cup M_8$\\
&&incongruent&$T_{mbi}=M_4 \cup M_6$\\
%\cmidrule{3-4}
&\multirow{2}{*}{adults'}
&congruent&$T_{mac}=M_1 \cup M_7 $\\
&&incongruent&$T_{mai}=M_3\cup M_5$\\
\cmidrule{1-4}
\cmidrule{1-4}
\multirow{4}{*}{non-mother}&\multirow{2}{*}{babies'}&congruent&$T_{nbc}=N_2\cup N_8$\\
&&incongruent&$T_{nbi}=N_4 \cup N_6$\\
%\cmidrule{3-4}
&\multirow{2}{*}{adults'}
&congruent&$T_{nac}=N_1\cup N_7$\\
&&incongruent&$T_{nbi}=N_3 \cup N_5$\\
\bottomrule
\end{tabular}
}
\end{table}


\noindent - \textbf{H3}:
Compared with when a non-mother being interfered by a facial expression from an adult,
an interfering facial expression from a baby could lead to a larger Stroop effect in the group of mother.
More formally,
according to the notations from Table~\ref{tab.hypothesis.3},
the Stroop effects over four groups are involved,
namely, 
 $T_{ma.}$, $T_{mb.}$, $T_{nb.}$, and $T_{na.}$.
 In addition to hypothesis 2, this hypothesis is summarized in
 Equation~\ref{eq.accuracy.h3} and~\ref{eq.rtime.h3}.

\begin{align}
\begin{split}\label{eq.accuracy.h3}
\mathit{accuracy}(T_{mbc})-\mathit{accuracy}(T_{mbi})&>
\mathit{accuracy}(T_{mac})-\mathit{accuracy}(T_{mai})\\
\mathit{accuracy}(T_{nbc})-\mathit{accuracy}(T_{nbi})&<
\mathit{accuracy}(T_{nac})-\mathit{accuracy}(T_{nai})\\
&\mathit{AND}\\
\end{split}\\
\begin{split}\label{eq.rtime.h3}
\mathit{rtime}(T_{mbc})-\mathit{rtime}(T_{mbi})&<
\mathit{rtime}(T_{mac})-\mathit{rtime}(T_{mai})\\
\mathit{rtime}(T_{nbc})-\mathit{rtime}(T_{nbi})&>
\mathit{rtime}(T_{nac})-\mathit{rtime}(T_{nai})\\
\end{split}
\end{align}

% Therefore, the mean of incongruent condition and mean of congruent condition are calculated first in each group. According to the \ref{tab.hypothesis.whole.design}, in the group of mothers, the reaction time and accuracy rate in the conditions m1, m2, m7, m8 are merged and examined as congruent conditions (is marked as condition m-c), and ones in the conditions m3, m4, m5, m6 are merged and examined as incongruent conditions (is marked as condition m-ic). Similarly, in the group of non-mothers the reaction time and accuracy rate in the conditions n1, n2, n7, n8 are merged and examined as congruent conditions  (is marked as condition n-c), and ones in the conditions n3, n4, n5, n6 are merged and examined as incongruent conditions (is marked as condition n-ic).


% Two means are then abstracted in each group, in order to get the Stroop effect of each group, namely, in the group of mothers, the reaction time and accuracy rate in m-c condition is abstracted by the ones in m-ic condition (the result is marked as ms). Meanwhile, in the group of non-mothers, the reaction time and accuracy rate in n-c condition is abstracted by the ones in n-ic condition (the result is marked as ns). Finally, we get two means (mstroop) and (nstroop), The hypothesis establishes, if two Stroop effect from the group of mother and from the group of non-mothers are different from each other(mstroop-nstroop not eual zero), in the direction that the Stroop effect in the group of mother is larger than the one in the group of non-mother (mstroop>nstroop).





% Compared with an interfering facial expression from an adult,
% a facial expression from a baby could impact more, 
% leading to a larger Stroop effect when participants face facial expressions of babies than when they face ones of adults.

% Similarly, the mean of incongruent condition and mean of congruent condition are calculated first separately in the condition of adults' faces and in the condition of babies' faces. According to the {tab.hypothesis.whole.design}, in the condition of adults' faces, the reaction time and accuracy rate in the conditions m1, n1, m7, n7 are merged and examined as congruent conditions (is marked as condition a-c), and ones in the conditions m3, n3, m5, m5 are merged and examined as incongruent conditions (is marked as condition a-ic). Similarly, in the condition of babies' faces the reaction time and accuracy rate in the conditions m2, n2, m8, n8 are merged and examined as congruent conditions  (is marked as condition b-c), and ones in the conditions m4, n4, m6, n6 are merged and examined as incongruent conditions (is marked as condition b-ic).

% Two means are then abstracted in each condition, in order to get the Stroop effect of each condition, namely, in the condition of adults' faces, the reaction time and accuracy rate in a-c condition is abstracted by the ones in a-ic condition (the result is marked as astroop). Meanwhile, in the condition of babies' faces, the reaction time and accuracy rate in b-c condition is abstracted by the ones in b-ic condition (the result is marked as bstroop). Finally, we get two means (astroop) and (bstroop), The hypothesis establishes, if two Stroop effect in the the condition of adults' faces and in the condition of babies' faces are different from each other (bstroop-astroop not eual zero), in the direction that the Stroop effect in the condition of babies' faces is larger than the one in the the condition of adults' faces (bstroop > astroop).






% The mean of incongruent condition and mean of congruent condition are calculated first separately in each four conditions \ref{tab.hypothesis.whole.design}.

% \begin{itemize}
% \item[(1)] in the condition combining the group of mother and adults' faces, the reaction time and accuracy rate in the conditions m1, m7 are merged and examined as congruent conditions (is marked as condition m-a-c), and ones in the conditions m3, m5 are merged and examined as incongruent conditions (is marked as condition m-a-ic).

% \item[(2)]  In the condition combining the group of non-mother and and adults' faces, the reaction time and accuracy rate in the conditions n1, n7 are merged and examined as congruent conditions  (is marked as condition n-a-c), and ones in the conditions n3, n5 are merged and examined as incongruent conditions (is marked as condition n-a-ic).

% \item[(3)]  In the condition combining the group of mother and and babies' faces, the reaction time and accuracy rate in the conditions m2, m8 are merged and examined as congruent conditions  (is marked as condition m-b-c), and ones in the conditions m4, m6 are merged and examined as incongruent conditions (is marked as condition m-b-ic).

% \item[(4)]  In the condition combining the group of non-mother and babies' faces, the reaction time and accuracy rate in the conditions n2, n8 are merged and examined as congruent conditions  (is marked as condition n-b-c), and ones in the conditions n4, n6 are merged and examined as incongruent conditions (is marked as condition n-b-ic).

% \end{itemize}

% Two means are then abstracted in each four conditions, in order to get the Stroop effect of each condition.
% \begin{itemize}

% \item[(1)]in the condition combining the group of mother and adults' faces, the reaction time and accuracy rate in m-a-c condition is abstracted by the ones in m-a-ic condition (the result is marked as mastroop). 
% \item[(2)]in the condition combining the group of non-mother and adults' faces, the reaction time and accuracy rate in n-a-c condition is abstracted by the ones in n-a-ic condition (the result is marked as nastroop).
% \item[(3)]In the condition combining the group of mother and babies' faces, the reaction time and accuracy rate in m-b-c condition is abstracted by the ones in m-b-ic condition (the result is marked as mbstroop).
% \item[(4)]In the condition combining the group of non-mother and babies' faces, the reaction time and accuracy rate in n-b-c condition is abstracted by the ones in n-b-ic condition (the result is marked as nbstroop).
% \end{itemize}
% The hypothesis establishes, if the Stroop effect in these four conditions are different from each other, more specifically the Stroop effect in the condition combining the group of mothers and babies' faces (nbstroop) is the largest than other three conditions.

% \end{itemize}
% \begin{table}[!t]
% \centering
% \caption{Experiment settings for the group of mother ($m1\cdots m8$,)  and non-mother ($n1\cdots n8$ ), indicating the number for the 8 different settings in each group.}
% \label{tab.hypothesis.3}
% \begin{tabular}{c|c|c|c|c}
% \toprule
% \multicolumn{3}{c|}{experiment settings}& adults' faces &babies' faces\\
% \midrule
% \multirow{4}{*}{mother}&\multirow{2}{*}{positive}&``Freude''&m1&m2\\
% &&``Ärger''&m3&m4\\
% \cmidrule{2-5}
% &\multirow{2}{*}{negative}
% &``Freude''&m5&m6\\
% &&``Ärger''&m7&m8\\
% \cmidrule{1-5}
% \cmidrule{1-5}
% \multirow{4}{*}{non-mother}&\multirow{2}{*}{positive}&``Freude''&n1&n2\\
% &&``Ärger''&n3&n4\\
% \cmidrule{2-5}
% &\multirow{2}{*}{negative}
% &``Freude''&n5&n6\\
% &&``Ärger''&n7&n8\\
% \bottomrule
% % \multirow{4}{c}{the group of non-mother}\\
% \end{tabular}
% \end{table}


%The combination of the mother and while facing the infant facial expressions is expected to be interfered most compared to other three combinations: mother facing the adult faces, non-mother facing the adult faces and non-mother facing the baby faces. 
%More specifically, on congruent trails the combination of mother and infant faces is expected to have the smallest reactions time, highest accuracy while on incongruent trails this combination is expected to have the largest reactions time, lowest accuracy and lowest quality.


%It is well-known that a preverbal infant communicates with its mother with emotional facial expressions, and a mother distinctly pays special attentions to such expressions and is well prepared to react spontaneously, e.g., to comfort a crying baby. Thompson-Booth and his colleagues (2014) looked close into such phenomenon and demonstrated the difference between a mother and a non-mother in their attentions in front of an emotional face from an infant and from an adult, where an attentional bias is observed toward the infant face. What has not been fully explored is that whether such difference also affects the ability of inhibitory control, namely, the ability to make certain reactions in terms of facial expressions in front of a displayed facial expression. Put differently, we would like to investigate whether a mother is more likely to be interfered by an emotional face from an infant in comparison to an adult.

%Existing works have demonstrated that there exist interference effects when an adult is given an emotional face (Dimberg, Thunberg, & Elmehed, 2000; Lunqvist, 1995; Wild, Erb, & Bartels, 2001). In particular, an adult is interfered by a displayed face when being asked to response with a certain facial expression, where the displayed face degrades his/her ability in fulfilling the tasks in terms of the time and the correctness. Recall that an emotional face from an infant could trigger stronger attentions from a mother than from a non-mother (Thompson-Booth et al., 2014). Therefore, one may expect such difference in attentions could also be mirrored in terms of the ability of inhibitory controls, namely, a mother is more likely to be interfered by an infant face. In order to examine such effects, an extended Stroop task is designed and implemented in this work, where a subject, either a mother or a non-mother, is instructed to react with a certain facial expression in front of an interfering facial expression from an adult or an infant as background. The time and the correctness of the reaction are measured. 

%Most importantly, we intend to find the interaction effect of motherhood and art of interference stimuli over the ability of inhibitory control, namely, a larger Stroop effect in mothers than non-mothers when facing baby facial expressions compared to adult ones is expected. Put differently, we expect a difference in resisting the interference of  production similar facial expression towards infant emotional faces between mothers and non-mothers in either positive (concordant emotions displayed by the face and the word) or negative way (discordant emotions displayed), where mothers would be more interfered by infant emotional faces. In a word, the Stroop effect is expected to be larger in mothers than in non-mothers when facing baby faces.


