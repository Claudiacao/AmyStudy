\chapter{Reinterpretation Outer-directed Character}\label{chp.reinterpretation}
As the criticisms and interpretations from the readers introduced in Chapter~\ref{chp.Criticism} are incomplete and relative pessimistic, in this chapter, the conformity of each three types of the social characters as well as the criticisms over the inner-directed type are introduced in the Section~\ref{chp.reinterpretation.conformity}. The descriptions of the other-directed type is then interpreted from a more objective and complete perspective, including the advantages in Section~\ref{chp.reinterpretation.advantage} and costs in Section~\ref{chp.reinterpretation.cost}.

\section{Conformity of Three Characters}\label{chp.reinterpretation.conformity}

In terms of conformity, the other-directed type of social character should not be specifically criticized, referring to the reasons stated in the following. 

% However, the conformity in the other-directed people, who use the peers and mass media to assess and understand how to act, to dress, and to behave, caused the most anxiety and was heatedly discussed.

First of all, such conformity tendency could be found in all three types of social character, which differs in to whom and how. Moreover, conformity is necessary and essential in forming the social character and in ensuring the normal functioning of the changing society. What David Riesman actually suggested was that, society could be thought in terms of a series of ``ideal types'' along a spectrum of increasingly loose authority and with different kinds of conformity, whereas, not conforming may lead to a a specific emotion corresponded to each social character.

\noindent - \textbf{Tradition-directed}\\
On one end of the spectrum is the ``tradition-directed'' community, where its members all understand that what they should to do is what they are supposed to do. Authority is unequivocal and there is neither the room nor the desire for autonomous action. Accordingly, tradition-directed people conform with the rules and regulations. The primary disciplining emotion under tradition direction is shame, the threat of ostracism and exile that enforces the traditional action.\\

\noindent - \textbf{Inner-directed}\\ 
In the middle of the spectrum, as one moves toward a freer distribution of the authority, is ``inner-direction''. The inner-directed person is concerned not with ``what one does'' but with ``what people tell me to do''. Put differently, he looks to his own internalizations of past authorities to get a sense for how to conduct his affairs, seeking after tangible goals, such as cars, house, wealth, which were the symbols of success, told by the gyroscopes planted by his parents. When not conforming, he experiences guilt instead of shame, and the fear that his behavior won’t be commensurate with the imago within.\\

\noindent - \textbf{Other-directed}\\
On the other end of the spectrum is the ``other-directed'' community, also known as the contemporary society, where the inculcated authority of the vertical (one’s lineage) gives way to the muddled authority of the horizontal (one’s peers), namely, its members look right and left instead of up and down. Compared to the inner-directed people, the other-directed one seek experiences over stuff, which could be seen as a better aim. However, their drive towards these experiences does not come from their own, but from watching other people. Therefore, the other-directed people do not just conform to others as far as external behavior, but also seek to match the quality of other people’s inner experience, looking to others for ``what experiences to seek and in how to interpret them''. In short, other-directed people conform with the thoughts (internal) and behaviors (external) from the mass media and peers. When not conforming, they experience a ``contagious, highly diffuse'' anxiety instead of guilty, and the possibility that they might be doing the wrong thing all the time, due to the fact that the authority itself is diffuse and ambiguous. 

That is to say, David Riesman drew no moral from the transition from a community of primarily inner-directed people to a community of the other-directed ones. Instead, he saw that each ideal type had different advantages and faced different problems. Moreover, when comparing the inner-directed member and other-directed one, whose characters are introduced in the~\ref{chp.background.characters}, the former does not consult some deep and subjective internal voices, instead, it consults the internalized voices of a mostly dead lineage, while the latter heeds the external voices of her living contemporaries. As is indicated by Riesman, ``the gyroscopic mechanism allows the inner-directed person to appear far more independent than he really is: he is no less a conformist to others than the other-directed person, but the voices to which he listens are more distant, of an older generation, their cues internalized in his childhood''. The inner-directed person is, simply, ``somewhat less concerned than the other-directed person with continuously obtaining from contemporaries (and the mass media) as a flow of guidance, expectation and approbation''. 

\section{Advantages of an Other-directed Type}\label{chp.reinterpretation.advantage}

Besides the not pleasant features of the other-directed type discussed in the Section~\ref{chp.Criticism}, we should not neglect the book’s careful emphasis on the positive aspects of it: openness, namely, interest in others as well as the flexibility, namely, the ability to change. 

\noindent - \textbf{Openness} describes a person who is intellectually curious, open to emotion, and willing to try new things. Accordingly, the other-directed person tends to be, when compared to inner-directed person, more creative and more aware of their feelings. Therefore, they tend to be politically liberal and tolerant of diversity~\citep{mccrae1996social,jost2006end}, making them more open to different cultures and lifestyles.\\  
\noindent - \textbf{Flexibility} is a personality trait that describes the extent to which a person can cope with changes in circumstances and think about problems and tasks in novel, creative ways~\citep{thurston1999flexibility}. The other-directed people, according to Riesman could adapt to situational demands, balance life demands, and commit to behaviors.
			
There is an old sentence in China saying that, when regarding a person as a mirror, you could find the mistakes and proper behavior of yourself. Due to the fact that there is no single guide book leading other-directed people to success, and there is different definitions of success, only by comparing with others, and continuously reflecting ourselves, one could find its own definition of success and go in this direction. However, comparing and going along with others do not mean simply copy others or follow others, it is important, to stay autonomous in this other-directed world, which will be discussed in Chapter~\ref{chp.discussion}.

\section{Cost of Being an Other-directed type}\label{chp.reinterpretation.cost}

Everything has its cost and the cost of virtues openness and flexibility introduced in Section~\ref{chp.reinterpretation.advantage} was a new form of anxiety about what to do and whom to trust, which is already introduced partly in Section~\ref{chp.reinterpretation.conformity}, presenting the greatest signal-to-noise-ratio problem in human history. Put differently, along with the transition from the inner-directed to the other-directed type, the modern society offers many new consuming possibilities, which is not a problem, instead, the pressures that peers and the
media used to foster socialized and often compulsive pleasure is. Under ideal conditions, modernization should liberate, instead of trap the individual~\citep{horowitz2010david}.  
			
For example, due to the lack of a unique definition of success, other-directed could easily be anxious, when seeing the different experiences that others are having. They look at Facebook and see a friend traveling the world, or partying in Vegas, or skydiving in South America, and wonder ``Is my life less satisfying?'', ``Should I be living more deeply than I am?'', ``Is everyone happier than I am?'' The resulting anxiety and restlessness could have a positive effect of motivating a man to get outside of his comfort zone and try new things, but it could also make him feel unhappy about his life choices, even if he made those choices willingly, consciously, and in line with his authentic desires, which could also keep him from making a choice he really wants, and just following others.









			
