\chapter{Introduction}\label{chp.introduction} 

%The introduction describes the research problem or research question and lays out the reasoning behind it.


%Why is it important to conduct the study?
%Include theoretical definitions of important terms and all constructs

It is well-known that the pregnancy may impact a female
emotionally and cognitively~\citep{jarrahi1969emotional}, 
as a result of the changes in a mother's\footnote{In this work,
a mother specifically refers to a biological mother
with a baby aging two to six months.}
brain and hormones
during both the pregnancy and the postpartum period~\citep{hoekzema2017pregnancy}.
In particular,
\citeauthor{hoekzema2017pregnancy} mentioned that
the volume of the gray matter  
in the prefrontal and temporal cortex decreases 
during pregnancy~\citep{hoekzema2017pregnancy}, 
resulting in changes in executive functions, in terms of 
the storage and processing of the information~\citep{de2006differences},
and of the attentional control~\citep{thompson2014here}. 
Meanwhile,
as one of the kernel executive functions, the ability of the inhibitory control is an ability to control one's attention, behavior, thoughts, and emotions, in resistance to impulses, established thoughts, as well as the stimuli from the environment~\citep{diamond2013executive}, which is influential to the speed of the information processing and the ability in resisting distractions. 


Consequently, 
one nature question to ask is that 
whether becoming a mother
could influence a female's ability of inhibitory
control, which, however, as far as our knowledge,
has not yet been well investigated.
To mitigate this gap,
in this work,
the ability of the inhibitory control of the mothers 
and of the non-mothers are compared, by inviting participants
to fulfill an extended Stroop task, where a participant is
asked to make a facial expression according to 
a one-word text instruction whereas being interfered
by a facial picture with the same or the opposing 
expression.
In addition,
given the special connection between a mother and a baby,
another question to ask is that whether an interference
involving a baby will influence a mother's ability of inhibitory
control differently.
% Another reason for that is
% it has been demonstrated that
% the face from a baby is more engaging compared with the one from
% an adult~\citep{glocker2009baby,luo2011children,sanefuji2007development}, we believe that a similar effect could also be found in the emotional facial expressions of babies and adults, namely, participants are more easily interfered by emotional facial expressions of babies compared to of adults.
Therefore,
the facial expressions from the babies are introduced
as interference, enabling to explore the interaction between a mother and the facial expression from a baby.
Meanwhile, as a comparison, the facial expressions from the adults
are also explicitly included in our experimental design.
To this end, 
we attempt to answer following
three research questions.
\begin{itemize}
\item[(a)] When compared with a non-mother, whether a mother is more prone to be interfered by an emotional facial expressions?
\item[(b)] When distinguishing the kinds of the
interfering facial expressions, namely, from a baby or from an adult,
whether the facial expression from the baby could interfere more?
\item[(c)] 
When jointly considering the impacts from the group of mother and group of non-mother, as well as from the two kinds of interfering expressions,
compared to a non-mother or to when being interfered by an adult's 
facial expression, 
whether a mother is more prone to be interfered by
a facial expression from a baby?
\end{itemize}

% summary of the method
To answer these three questions,
inspired by~\citep{otte2011interference,lee2007controlling}, 
the extended Stroop task is first designed
to incorporate the emotional interference in the Stroop task,
particularly tailoring to this study,
where
a participant is instructed to fulfill the tasks 
under
certain interfering configurations.
% Different comparisons in the aforementioned 
The mentioned research questions are 
investigated via the different designs of the 
interfering conditions.
% where different conditions of 
% the congruency between
% the interfering expression and the word, as well as
% the different sources of 
% the facial expressions are configured. 
Meanwhile, 
likewise in~\citep{otte2011interference,lee2007controlling},
the responses from the participants are recorded, on which
the electromyography (EMG) is used to measure the accuracy and the response time, both providing concrete measures of 
the ability of the inhibitory control
(namely, the impacts of the interferences).
Put differently, a higher accuracy and a shorter response time 
are assumed to indicate that a participant 
enjoys a better ability of the inhibitory control and vice verse.
Accordingly, akin to~\citep{bugg2008multiple}, the Stroop effects are
measured in terms of the differences among
the observed ability of the inhibitory control when the participants are exposed to different interfering conditions.
\footnote{This study is part of the 
\href{https://www.psychologie.hu-berlin.de/de/prof/bio/forschung/forschungsprojekte}{the parenthood project}
supported by Deutsche Forschungsgesellschaft
at Psychology Institute of Humboldt University. }  


% summary of the conclusion
Through our analyzes,
when independently comparing a mother with a non-mother,
or comparing the interfering expressions from
a baby and an adult,
there is no evidence to support that 
a mother has a lower ability of 
inhibitory control than a non-mother in general,
nor does a baby's facial expression interfere more,
opposing to the findings of the studies~\citep{lorenz1943angeborenen,glocker2009baby,luo2011children,sanefuji2007development}~that a baby face is more attractive and therefore
must be more distractive.
When jointly considering both factors, however,
the observed Stroop effect 
from a mother differs when she is interfered 
by the facial expressions from a baby.
Namely,
a non-mother is more prone to 
be interfered by
an adult, 
meanwhile a mother responses uniformly to either kind,
but tends to be interfered more by a positive facial expression from a baby.
In other words,
though a smiling makes no difference when considering mothers and non-mothers together,
it does distract a mother more.
% summary of the contribution
As a pilot study, the contributions of this work are threefold: 
(1) An extended Stroop task is designed to incorporate the emotional interference on top of the traditional Stroop task;
(2) Under the parenthood study project, a significant amount 
of empirical data is collected, enabling this work, as well as 
the future works on similar topics;
and (3) Through intensive analyzes, 
the impacts of the 
interaction between a mother and a baby over the ability
of the inhibitory control are investigated and highlighted.


% In the task, as mentioned,
% a participant is asked to make a facial expression according to a word, which is 
% isplayed together with an interfering picture including an emotional facial expression.In particular, a participant is asked to simile or frown 
% in response to a word, namely, 
% either ``happy'' or ``angry'', 
% meanwhile a photo including a facial expression with 
% the same or different emotion, 
% denoted as ``positive'' and ``negative'' respectively,  serves to interfere. 
% To further encode the concept of motherhood,
% in this study,the interfering facial expressions include pictures from 
% both adults and babies.
% For example, as displayed in Figure~\ref{fig.method.task},
% a word ``happy'' as well as an angry facial expression from a baby are displayed, 
% where a smile is expected from the participants 
% as a correct response
% according to the word ``happy''.







% To this end, we first examine the influence of the motherhood and the influence of the identity of interfering facial expressions upon the ability of inhibitory control separately; furthermore, the interactions effect between them is investigated. 
%  Akin to  Electromyography (EMG) is employed to measure the accuracy and the response time of the facial expression from a subject by capturing the movements of two groups of facial muscles, namely, the zygomatic major muscle controlling smiling and the corrugator supercilii muscle controlling frowning. 


% % summary of the conclusion
% According to the data, there is no evidence to support that the status of being a mother or a non-mother or the identity of a facial expression alone impacts the ability of inhibitory control. However, when further considering the types of the interference, namely, a positive or a negative facial expression, we do find that a mother is more prone to a positive interference compared with a non-mother. Besides, we also find that negative facial expressions of adults have larger impact compared with ones of babies. Moreover, in consistence with our third hypothesis, the non-mothers are found more prone to be interfered by the facial expressions from the adults, whereas the mothers seem to behave similarly in front of different kinds of interferences. When further considering the types of the interference, we can conclude that the mothers' ability of inhibitory control actually depends on the property of the interference, namely,in front of a positive (smile) interference,the mothers are influenced more by the babies;
% whereas the adults influence more when a negative 
% interference (frown or cry) is given. The significant effect in the group of non-mother could be explained by the emotion expression interference (EEI) effect, defined by \citep{lee2007controlling}, indicating that when viewing a facial expression, expressing a
% different emotion would manifest as behavioral conflict and interference. A facial expression from an adult is expected to be larger than from a baby, since people is tuned to be sensible in noticing and reacting to the facial expressions from other adults, leading to a lower ability of inhibitory control
% in front of the facial expressions from other adults; 
% Meanwhile, as for the mothers, stem from the biological changes, they pay special attentions to facial expressions of babies and are well prepared to react spontaneously with facial expressions, making their ability of inhibitory control differ from the ones of the non-mothers.

% % summary of the contribution
% The contributions in this work are twofold: 1) A extended Stroop task is designed to incorporate the emotional interference in the traditional non-emotional Stroop task 2) The interactions effect between the status of the motherhood and the identity of the interfering facial expressions over the inhibitory control of resisting the emotional interference is found and possible explanations are discussed.







% organizations of the remaining
The remaining chapters are sketched as follows.
Chapter~\ref{chp.background} summarizes the exiting literatures
and put our work in context,
where
the influence of motherhood is introduced in Section~\ref{sec.influenceofmother},
the ability of the inhibitory control is summarized in Section~\ref{sec.abilityofinhibitorycontrol},
and the measurement of the facial expressions 
is described in Section~\ref{sec.facialexp}. 
Thereafter,
Chapter~\ref{chp.hypothesis} 
introduces the design of the extended Stroop task in Section~\ref{sec.hypothesis.expstrooptask}, 
following which the 
experiment design and three hypotheses are discussed in details in Section~\ref{sec.hypothesis.design} and~\ref{sec.hypothesis}. 
To examine these hypotheses, 
the methods employed are summarized in 
Chapter~\ref{chp.method},
where the experimental setup, the procedure of the experiment 
and the data processing procedures are introduced in details 
in Section~\ref{sec.method.setup}, \ref{sec.method.details} and~\ref{sec.method.postprocessing} respectively. 
Before conclusion,
Chapter~\ref{chp.results} presents the results including an overview in Section~\ref{sec.results.overview} and the detailed tests of the hypotheses in Section~\ref{sec.results.hypothesis}.
In the end,
Chapter~\ref{chp.discussion} concludes 
this thesis in Section~\ref{sec.discussion.conclusion},
and discusses the limitations and the future works
in Section~\ref{sec.discussion.furtherwork}.


% the identity of the interfering facial expressions, namely a facial expression from a baby or an adult, is also expected to have an influence over the ability of inhibitory control. 
% More specifically, in our study, the ability of resisting the instinct of making similar facial expressions to the facial expressions of the interferences is measured. 
% In short, we expect that a mother is more prone to be influenced by the interferences in terms of emotional facial expressions.












% However, as one of the core executive functions, the influence of status of a mother upon the ability of inhibitory control, namely, upon the ability of controlling one's attention, behavior and emotions, has not been investigated. 






% It is well-known that a preverbal infant communicates with its mother with emotional facial expressions, and a mother distinctly pays special attentions to such expressions and is well prepared to react spontaneously with facial expressions.
% \citeauthor{thompson2014here} investigates the influence of the motherhood and the facial expressions from babies over the attentional allocation, finding that mothers are more easily being attracted by emotional faces of babies compared to non-mothers \citep{thompson2014here}. However, existing works have also demonstrated that there exist interference effects when an adult is given an emotional face of adults~\citep{dimberg2000unconscious,lundqvist1995facial,wild2001emotions}. In particular, an adult is interfered by a displayed adult face when being asked to response with a certain facial expression, where the displayed face degrades his/her ability in fulfilling the tasks in terms of the time and the correctness~\citep{lee2007controlling,otte2011interference}. Recall that an emotional face from an infant could trigger stronger attentions from a mother than from a non-mother ~\citep{thompson2014here}. Therefore, one may expect such difference in attentions could also be mirrored in terms of the ability of inhibitory controls, namely, a mother is more likely to be interfered by an emotional facial expression of a baby.

% To this end, we first examine the influence of the motherhood and the influence of the identity of interfering facial expressions upon the ability of inhibitory control separately; furthermore, the interactions effect between them is investigated. To incorporate the emotional interference in a Stroop task,
% inspired by the emotional interference tasks~\citep{otte2011interference,lee2007controlling}, 
% an extended Stoop task is proposed.
% According to~\citep{dimberg2000unconscious,lundqvist1995facial,wild2001emotions},
% the procedure to simulate a facial expression (with a facial expression)
% is also ``automatic process'',
% whereas a facial expression in response to a word is not habitual. A participant is asked to make a facial expression according to a word, which is 
% displayed together with an interfering picture including an emotional facial expression. 
% In particular, a participant is asked to simile or frown 
% in response to a word, namely, 
% either ``happy'' or ``angry'', 
% meanwhile a photo including a facial expression with 
% the same or different emotion, 
% denoted as ``positive'' and ``negative'' respectively, 
% serves to interfere. 
% To further encode the concept of motherhood,
% in this study,
% the interfering facial expressions include pictures from 
% both adults and babies.
% For example, as displayed in Figure~\ref{fig.method.task},
% a word ``happy'' as well as an angry facial expression from a baby are displayed, 
% where a smile is expected from the participants 
% as a correct response
% according to the word ``happy''.
% the Stroop effects are in terms of the different responses when facing congruent or incongruent interference,
% namely, a shorter (longer) response time and a higher (lower) accuracy in response to a congruent (incongruent) pair of word and facial picture.
% Therefore, akin to~\citep{bugg2008multiple}, in this work, the Stroop effect is quantified in terms of the differences 
% of the response time and of the accuracy between when presenting a congruent and an incongruent interferences. Akin to \citep{otte2011interference,lee2007controlling}, Electromyography (EMG) is employed to measure the accuracy and the response time of the facial expression from a subject by capturing the movements of two groups of facial muscles, namely, the zygomatic major muscle controlling smiling and the corrugator supercilii muscle controlling frowning. 

% 

% The contributions in this work are twofold: 1) A extended Stroop task is designed to incorporate the emotional interference in the traditional non-emotional Stroop task 2) The interactions effect between the status of the motherhood and the identity of the interfering facial expressions over the inhibitory control of resisting the emotional interference is found and possible explanations are discussed.





% It is well-known that a preverbal infant communicates with its mother with emotional facial expressions, and a mother distinctly pays special attentions to such expressions and is well prepared to react spontaneously, e.g., to comfort a crying baby. Thompson-Booth and his colleagues (2014) looked close into such phenomenon and demonstrated the difference between a mother and a non-mother in their attentions in front of an emotional face from an infant and from an adult, where an attentional bias is observed toward the infant face. What has not been fully explored is that whether such difference also affects the ability of inhibitory control, namely, the ability to make certain reactions in terms of facial expressions in front of a displayed facial expression. Put differently, we would like to investigate whether a mother is more likely to be interfered by an emotional face from an infant in comparison to an adult.

% Existing works have demonstrated that there exist interference effects when an adult is given an emotional face (Dimberg, Thunberg, & Elmehed, 2000; Lunqvist, 1995; Wild, Erb, & Bartels, 2001). In particular, an adult is interfered by a displayed face when being asked to response with a certain facial expression, where the displayed face degrades his/her ability in fulfilling the tasks in terms of the time and the correctness. Recall that an emotional face from an infant could trigger stronger attentions from a mother than from a non-mother (Thompson-Booth et al., 2014). Therefore, one may expect such difference in attentions could also be mirrored in terms of the ability of inhibitory controls, namely, a mother is more likely to be interfered by an infant face. In order to examine such effects, an extended Stroop task is designed and implemented in this work, where a subject, either a mother or a non-mother, is instructed to react with a certain facial expression in front of an interfering facial expression from an adult or an infant as background. The time and the correctness of the reaction are measured. 

% Intuitively, such effects could be essential in raising a baby and might stem from the changes of neuroendocrine systems when a female becomes a mother. Inspired by this insight, we also explore the reasons for such behaviors from the perspective of neuroendocrine systems. Actually, a cascade changes in neuroendocrine systems, especially the hormone, is associated with the pregnancy and childbirth, and is believed helpful in regulating maternal behavior. Accordingly, in this study, we select two hormones, namely, the oxytocin and the cortisol. The former is well-known in its influences in social cognitions, especially in the mother-child relationship, and a higher amount is observed from a mother compared with from a non-mother; while the latter one is related to the level of stress, opposing to the effects of oxytocin (Krause et al., 2016; Gordon et al., 2008; McQuaid et al., 2016), where, however, a rising level of cortisol might be observed due to the stressful baby sitting, compromising the rising of oxytocin.

% To this end, we first examine the effects of motherhood upon the ability of inhibitory control when facing a facial expression from a baby by designing an extended Stroop task; furthermore, we also investigate the relationship between two particular hormones, which are related to baby-care and stressfulness, attempting to explain the observations from the extended Stroop task. The contributions in this work are twofold: 1) The videos are employed in measuring the facial responses, which not only countercheck the results from EMG but also enable a third comparison, namely, measuring the quality of the facial expressions; 2) The hormone is introduced to explain the observed phenomenon.

% Our study intends to reveal that whether the motherhood could influence the ability of inhibitory control, in other words, whether a mother is more likely to be interfered by an emotional face from an infant in comparison to an adult. The combination of the mother and while facing the infant facial expressions is expected to be interfered most compared to other three combinations: mother facing the adult faces, non-mother facing the adult faces and non-mother facing the baby faces. This interference effect has two directions, positive (concordant emotions displayed by the face and the word) or negative (discordant emotions displayed), which could be measured by the reaction time, accuracy as well as the quality of the facial expressions from the subject. More specifically, on congruent trails the combination of mother and infant faces is expected to have the smallest reactions time, highest accuracy and highest quality while on incongruent trails this combination is expected to have the largest reactions time, lowest accuracy and lowest quality.

% Method

% Stroop task is one of the widely used neuropsychological tests to measure the interference. In an original Stroop task, a subject is asked to name the ink color of a word, which, literally, also denotes a color. Therefore, a different color denoted by the word could interfere the reaction from the subject, e.g., "red" in green ink. The Stroop effects indicate that a subject might take longer and is more prone to mistakes when differing colors are given from the text and from the ink, as a result of the spontaneity in understanding the text than in recognizing the ink color. The theory of response automaticity explains such difference using the automatic process of habitual reading which involves no controlled attentions whereas the recognition of colors does. Thus, the automatic response to the text interferes the desired response to the color of the ink. 
% Likewise, several studies reveal that when people are exposed to facial expressions they spontaneously react to these expressions with similar facial expressions (Dimberg, Thunberg, & Elmehed, 2000; Lundqvist, 1995; Wild, Erb, & Bartels, 2001), which are based on automatic processes (Dimberg, Thunberg, & Grunedal, 2002), meanwhile making a facial expression according to a word is not automatic and requires controlled attentions. 
% Therefore, mirroring the original color-word Stroop task, the photos of facial expressions could interfere the desired response for a given word. In particular, in our study, a subject is asked to simile or frown in response to a word, namely, either "happy" or "angry", meanwhile a photo including a facial expression bearing the same or different emotion, denoted as "positive" and "negative" respectively, is employed as an interfering source. Furthermore, to study the effects of motherhood, the interfering photos are either from an infant or from an adult. For example, in our extended Stroop task, the word "happy" as well as an angry facial expression from a baby are displayed, where a subject should smile according to the instruction。

% Akin to Otte et al. (2010), Electromyography (EMG) is employed to measure the correctness and the response time (RT) of the facial expression from a subject by capturing the movements of two groups of facial muscles, namely, the zygomatic major muscle controlling smiling and the corrugatorsuperciliimuscle controlling frowning. In addition to EMG, we further employ videos for measurement by recording the facial expressions from a subject. On one hand, the results from EMG could be counterchecked by comparing with the ones from video; on the other hand, a third metric, namely, the quality of the facial expression, is introduced. The influences of motherhood, in either positive (concordant emotions displayed by the face and the word) or negative way (discordant emotions displayed), over the ability of inhibitory control are examined in terms of the comparison of different measures, e.g., there might be a longer RT, a lower accuracy, as well as a lower quality of the responding facial expression when infant photos and words displaying discordant emotions are provided to the subjects. 

% Statistics are collected in terms of response time and correctness extracted from EMG and Video recording. In an ANOVA test, facial identity (faces from an adult vs. from a baby), words (“Happy” vs. “Angry”), and facial expressions (faces with positive vs. negative emotions) are used as three within-subject factors where a subject participates the extended Stroop task including different combination of the mentioned factors, namely, a mother (non-mother) takes a Stroop task covering all eight situations (Figure 1) corresponding to the eight cells (Table 1) under the column named Mother (Non-Mother); meanwhile the motherhood is employed as between-subject factor.





