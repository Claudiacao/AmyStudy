\chapter{Solution}\label{chp.solution}

As the most part of our thoughts and behaviors are other-directed in the modern society, it is worth to know that whether it is possible to have a trade-off of the advantages and costs of the other-directed character, put differently, whether it is possible to be less conform and anxious, and at the meantime enjoy the virtues, including openness and flexibility. At the end of book ``The Lonely Crowd'', David Riesman offered a solution, where an ideal type of social character is introduced, namely, a fourth type: the autonomous, which is original defined as the capacity of a person to make an informed, un-coerced decision.

According to the descriptions in the book, the autonomous person has ``clear cut, internalized goals'', but unlike the inner-directed one, he chooses those goals for himself, which are ``toward them, rational, non-authoritarian and not compulsive''. He is capable of cooperating with others like the other-directed, but ``maintains the right of private judgment''. Put differently, he is involved in his world, but his ``acceptance of social and political authority is always conditional''.

In terms of the conformity, the autonomous people ``are those who on the whole are capable of conforming to the behavioral norms of their society but are free to choose whether to conform or not'', leading to the standout and superiority of this type of social character to the other types, namely, the autonomous people understand the other three types of social character, being able to reflect on them and then freely choosing when and if to resist them or act in accordance with them. 

What kind of a person is the autonomous man in a predominately other-directed society likely to be in the future? This is a question which the authors treat fleetingly in the book. In a review of The Lonely Crowd, Richard L. Meier and Edward C. Banfield suggested that ``the new autonomous type will be very much affected by the tremendous quantities of information that are open to him, and by the comprehensive quick-acting, and relatively unbiased institutions which he can use. His relationship to the machine will be that of designer or diagnostician, but not slave. His logic will be multi-valued, often with concrete statistical formations.'' To conclude, the autonomous type of social character could be a very good solution in reducing the anxiety and strong conformity in the other-directed type, thereafter is pragmatic but also idealistic at the same time, due to many challenges and difficulties during the realization, which will be discussed in~\ref{chp.discussion}. Still, it is an ambitious goal, which worth striving for.



