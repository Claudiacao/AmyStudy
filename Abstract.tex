\addchap{Abstract}

It has been demonstrated that 
becoming a mother
could trigger physical and   
psychological changes.
Meanwhile, 
the ability of inhibitory control could influence one's
behaviors when in front of the interferences, namely,
being more or less capable in resisting distractions.
However, it has not been fully investigated 
whether becoming a mother could change the ability
of the inhibitory control.
To bridge this gap,
in this work, 
the motherhood and the ability of inhibitory control 
are co-investigated in an empirical study
\footnote{This study is part of the 
\href{https://www.psychologie.hu-berlin.de/de/prof/bio/forschung/forschungsprojekte}{the parenthood project}
supported by Deutsche Forschungsgesellschaft
from the Department of Biological Psychology and Psychophysiology led by Prof. Werner Sommer at the Psychology Institute in Humboldt University.}.
In particular, 
an extended Stroop Task is  
designed to quantify the ability of inhibitory control
in front of different emotional interferences;
thereafter,
a group of mothers and non-mothers are invited to 
fulfill the task under different interfering configurations, whose responses are assessed
using electromyography (EMG);
ultimately,
their performances, in terms of accuracy and response time,
under different configurations are
compared to draw conclusions.
Compared with a non-mother,
the experiment demonstrates that 
a mother differs in her ability of inhibitory control
when being interfered by a baby.