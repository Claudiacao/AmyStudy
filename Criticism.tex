\chapter{Criticisms on Other-directed and its Conformity}\label{chp.Criticism}

The transition from the inner-directed type of society to the other-directed type with a ``stronger'' conformity has led to the discussions and concerns of many American readers. In this chapter, the descriptions of the conformity in the other-directed people in the book are introduced in Section~\ref{chp.Criticism.Book}, resulting in the interpretation and criticisms over the other-directed type of society, which are introduced in Section~\ref{chp.Criticism.Readers}.

\section{Conformity of Other-directed In the Book}\label{chp.Criticism.Book}

Conformity is the extent to which an individual complies with group norms or expectations. Erich Fromm defined the ``automation conformity'' as a mechanism of a human being in his work ``The Fear of Freedom'', which could be seen as an extreme case of the conformity. Having the automation conformity, an individual ceases to be himself and adopts entirely the kind of personality offered to him by cultural patterns, becoming exactly as all others are and as they expect him to be. This mechanism can be compared with the protective coloring some animals assume, looking so similar to their surroundings, being hardly distinguishable. As a result, the discrepancy between ``I'' and the world, with the conscious fear of aloneness and powerlessness disappears, namely, the person who gives up his individual self and becomes an automaton, identical with millions of other automatons around him, does not need to feel alone and anxious any more. But the price he pays, however, is high, namely, the loss of his self~\citep{fromm1941fear}.

Recall the descriptions in the book ``The Lonely Crowd'', other-directed people are easily influenced by the peer groups and mass media~\ref{chp.background.characters}, which induce them to conform due to the peer pressure, whose influence is not only external, such as dress, act but also internal, such as thoughts and emotions. Similarly to the description of the automation conformity~\citep{fromm1941fear}, the conformity of the other-directed people also leads to following the trend, namely, wearing a uniform, belonging and thinking uniform thoughts. Because they are afraid of being different, standing out or taking a stand on important issues. Thereafter the group of contemporaries, comrades and colleagues play a very important role in their life. Different from the disappear of the aloneness and powerlessness after the conformity, indicated by Erich Fromm~\citep{fromm1941fear}, David Riesman suggested that having lost touch with themselves, other-directed souls were actually alone in the midst of other people, therefore, as donated from the name of the book, feel lonely. Moreover, society dominated by the other-directed people also faces profound deficiencies in leadership, individual self-knowledge, and human potential. 

\section{Interpretation and Criticisms by Readers}\label{chp.Criticism.Readers}
Along with Erich Fromm’s ``Escape from Freedom''~\citep{fromm1941escape}, William H. Whyte Jr.’s ``The Organization Man''~\citep{whyte2002organization} and Vance Packard’s ``The Status Seekers''~\citep{packard1959status}, most of the reading public and many critics saw ``The Lonely Crowd'' more narrowly, who links the autonomy and inner-direction closely, as ``a great secular jeremiad against other-direction''~\citep{mcclay1998lonely} and as part of the social pessimism which is fashionable in this period, arguing that the society is heading in the wrong direction~\citep{steenvoorden2016societal}. 

Due to the fast development of the social networks, such as Facebook and Twitter, more and more people worry about how mass media threatens individualism by promoting conformity. For example, children no longer care much about the authority of adults but are rather hyperalert to the peer groups and gripped by mass media, which is sometimes incorrect and misleading. ``Father might know best, but if he did, it was increasingly because a television program said so.'' At the meantime, the adults read the commends of an affair on the newspaper or in the Internet, then chatted with their colleagues without thinking on their own. 



