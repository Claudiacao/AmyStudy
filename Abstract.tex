\addchap{Abstract}

The concept ``social character'', first defined by Erich Fromm, describes the common emotional attitudes and psychological reactions to people in a social class or in a society~\citep{fromm1941escape}. 

inspired by the concept, David Riesman wrote the book ``The Lonely Crowd'', who is a sociologist and a social critic as well, and has a
major influence on American cultural life. In this book, three social characters based on the American society are identified, where individuals are classified as the members of different characters. Namely, the tradition-directed members obey the rules; the inner-directed members are akin to follow the inner gyroscope; and the other-directed members tend to go after the peers and mass media~\citep{riesman2001lonely}. 

Among this three categories, the other-directed members, who are particularly influenced externally rather than internally, are often criticized for their conformity~\citep{mcclay1998lonely}. In this article, I argue that some of such criticisms are not objective enough and tend to interpret the conformity pessimistically. To bridge the gap, the concept of the conformity is first introduced and then analyzed in the context of each three social character. Besides, some positive aspects of the other-direct character are highlighted as well as their costs. Thereafter, the solution for the conformity, a fourth social character of the changing society, namely, the autonomous, is introduced, where the difficulty and feasibility to achieve the autonomy in the other-directed world are further discussed. 


