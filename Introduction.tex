\chapter{Introduction}\label{chp.introduction}

As a theoretical sociologist, Erich Fromm first defined the concept ``social character'' by integrating Marx's theory concerning how the mode of production determines ideology with Freud's concept of character. Instead of the common emotional attitudes and psychological reactions to people of an individual, ``social character'' is defined as ones in a social class or society~\citep{fromm1941escape}. 

Influenced by Erich Fromm personally and intellectually, David Riesman considered the concept ``social character'' in the context of the American society and identified three types of social characters in his book ``The Lonely Crowd'', first published in 1950 as a sociological analysis of American life. Surprisingly, it soon became a bestseller and is considered by many to be the most influential book of the twentieth century, resulting in the concerns of people all over the world, especially of Americans, about the changing of the social character.

The three types of the social characters based on the American society are the tradition-directed, the inner-directed and the other-directed type. Additionally, the transitions between them are also introduced in this book~\citep{riesman2001lonely}. The society with high birth rate and high death rate was dominated by the tradition-directed type, whose members tend to stick to the rules. As the death rate declines, the social character shifted from the inner-directed type to the other-directed one, whose members akin to behave according to their own inner gyroscope implanted by elders. As the birth rate began to follow the death rate downward, the social character shifted from the inner-directed type in 19th-century to the other-directed type in the mid-20th-century, whose members become very sensitive to the preferences and expectations of others.

The intention of ``The Lonely Crowd'' was primarily to analyze the American life rather than to point with anxiety to its deficiencies, however, as the sociologist Dennis H. Wrong observed, ``it was widely read as deploring the rise of the psychological disposition called `other-direction' at the expense of `inner-direction'. '' The conformity of the other-directed people, who have little autonomous and seem to follow the trend all the time, is often criticized by the readers. 

%As the source of ``evil'', the culture are severe criticized as well, especially the informative and interactive social media, inducing the public to become more other-directed and to go conformity, while misleading and disorienting them at the same time.

However, the conformity is actually necessary for the changing society, ensuring the normal functioning of it. Each of the three types of social characters has the conformity trend, differing in to whom and how, which is not unique in the other-directed type. Moreover, the other-directed member seems to be more autonomous than the other-directed person, but is no less a conformist to others, since they are akin to listen to the old generations.

As economists often say, everything has a price. The cost of the transition from an inner-directed type to an other-directed type is not so bad as many thought. The positive aspects of the other-direct character, namely, openness and flexibility should not be overlooked, although it may lead to the anxiety about what to do and whom to trust.

In terms of the concerns about the conformity, the solution for the conformity, a fourth social character of the changing society, namely, the autonomous is introduced, whose members are capable of conforming to the behavioral norms of their society but are free to choose whether to conform or not. However, there are many challenges to become an autonomous person, for example, the anxiety when we face the uncertainty, and the impact of the mass media, making the conformity much easier and reasonable. To achieve autonomy in an other-directed world, one should have confidence in himself as well as the self-awareness, while think critically to the large amounts of information.
 
The chapters are sketched as follows. The work begins with the background in Chapter~\ref{chp.background} including the introduction of the main author, David Riesman, and his work ``The Lonely Crowd'' in Section~\ref{chp.background.david} as well as the definitions of three types of the social characters in Section~\ref{chp.background.characters}. In Chapter~\ref{chp.Criticism}, the statements about the other-directed type in the book are listed in~\ref{chp.Criticism.Book}, while in the Section~\ref{chp.Criticism.Readers}, some well-known critics about the conformity of other-directed people from readers are introduced. In Chapter~\ref{chp.reinterpretation}, the conformities of each three characters are introduced in Section~\ref{chp.reinterpretation} at the very beginning, indicating that the conformity does not exist only in the other-directed type. Thereafter, the advantages and the costs of being a other-directed type are introduced respectively in Section~\ref{chp.reinterpretation.advantage} and~\ref{chp.reinterpretation.cost}. In Chapter~\ref{chp.solution}, a solution in terms of the conformity of the other-directed type is introduced, namely, autonomous. In Chapter~\ref{chp.discussion}, the work is summarized and the autonomous is discussed in the context of the modern society.
